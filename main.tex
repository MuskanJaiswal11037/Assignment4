
\usepackage[utf8]{inputenc}
\documentclass{beamer}
\usetheme{CambridgeUS}
\usepackage{listings}
\usepackage{blkarray}
\usepackage{listings}
\usepackage{subcaption}
\usepackage{url}
\usepackage{tikz}
\usepackage{tkz-euclide} % loads  TikZ and tkz-base
%\usetkzobj{all}
\usetikzlibrary{calc,math}
\usepackage{float}
\renewcommand{\vec}[1]{\mathbf{#1}}
\usepackage[export]{adjustbox}
\usepackage[utf8]{inputenc}
\usepackage{amsmath}
\usepackage{amsfonts}
\usepackage{tikz}
\usepackage{hyperref}
\usepackage{bm}
\usetikzlibrary{automata, positioning}
\providecommand{\pr}[1]{\ensuremath{\Pr\left(#1\right)}}
\providecommand{\mbf}{\mathbf}
\providecommand{\qfunc}[1]{\ensuremath{Q\left(#1\right)}}
\providecommand{\sbrak}[1]{\ensuremath{{}\left[#1\right]}}
\providecommand{\lsbrak}[1]{\ensuremath{{}\left[#1\right.}}
\providecommand{\rsbrak}[1]{\ensuremath{{}\left.#1\right]}}
\providecommand{\brak}[1]{\ensuremath{\left(#1\right)}}
\providecommand{\lbrak}[1]{\ensuremath{\left(#1\right.}}
\providecommand{\rbrak}[1]{\ensuremath{\left.#1\right)}}
\providecommand{\cbrak}[1]{\ensuremath{\left\{#1\right\}}}
\providecommand{\lcbrak}[1]{\ensuremath{\left\{#1\right.}}
\providecommand{\rcbrak}[1]{\ensuremath{\left.#1\right\}}}
\providecommand{\abs}[1]{\vert#1\vert}

\newcounter{saveenumi}
\newcommand{\seti}{\setcounter{saveenumi}{\value{enumi}}}
\newcommand{\conti}{\setcounter{enumi}{\value{saveenumi}}}
\usepackage{amsmath}
\setbeamertemplate{caption}[numbered]{}

\title{Assignment -4}       
\author{Muskan Jaiswal -cs21btech11037}
\date{May 2022}
\logo{\large \Latex{}}
\begin{document}
\begin{frame}
		\titlepage
	\end{frame}

\begin{frame}{Outline}
  \tableofcontents
\end{frame}

\section{Abstract}
	\begin{frame}{Abstract}
		\begin{itemize}
			\item 	This document contains the solution to Question of Chapter 6 (Probability) in the NCERT Class 12 Textbook.
		\end{itemize}
	\end{frame}



\section{QUESTION }
\begin{frame}{}
\begin{block}{}
    In a meeting, 70\% of the members favour and 30\% oppose a certain proposal. A member is selected at random and we take X=0 if he opposed, and X=1 if he is in favour. Find 1.E(X) and 2.Var(X).
\end{block}
\end{frame}
\section{Answer}
\begin{frame}{}
According to the question, X be  the random variable where\\
X=0, if the member oppose a certain proposal\\
X=1, if the member favours a certain proposal\\
\subsection{1^{st} part :}
\begin{table}[]
    \centering
    \begin{tabular}{|c|c|}
        \hline
        X &P(X)  \\
        \hline\hline
        0 & $\frac{3}{10}$\\
        \hline
        1 & $\frac{7}{10} $\\
        \hline
    \end{tabular}
    

\end{table}

\begin{align}
E(X)=&\sum_{i=1}^{n}x_i P(x_i)\\
E(X)=&0\times \frac{3}{10}+1\times \frac{7}{10}\\
E(X)=&\frac{7}{10}
\end{align}
\end{frame}
\begin{frame}{}
\subsection{2^{nd} part :}

\begin{align}
Var(X)=&E(X^2)-(E(X))^2\\
E(X^2)=&\sum_{i=1}^{n}{x_i}^2 P(x_i)\\
E(X^2)=&{0}^2\times \frac{3}{10}+{1}^2\times \frac{7}{10}\\
=&\frac{7}{10}\\
Var(X)=&E({X}^2)-({E(X)})^2\\
=&\frac{7}{10}-\frac{{7}^2}{{10}^2}\\
=&\frac{7}{10}-\frac{49}{100}\\
=&\frac{21}{100}\\
\end{align}

\end{frame}
\end{document}
